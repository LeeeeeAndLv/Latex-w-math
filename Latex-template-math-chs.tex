% Latex template with some math formulas in English 

\documentclass{article}
\usepackage[slantfont, boldfont]{xeCJK}
\usepackage{fullpage,amsmath,amsthm,amssymb}

% Set English main font
\setmainfont[ItalicFont={Times New Roman}]{Times New Roman}

% Set Chinese main font 
% In order to compile correctly, the font should be included in your operation system
\setCJKmainfont[ItalicFont={FangSong}, BoldFont={SimHei}]{SimSun}
\setCJKmonofont{SimSun}
\setCJKsansfont{FangSong}
\renewcommand\tablename{表}% Change table name
\renewcommand\figurename{图}% Change figure name
\renewcommand\refname{参考文献}
%\renewcommand\bibname{参考文献} % for a book


\newtheorem{thm}{定理}
\newtheorem{rmk}{注记}[section]

\title{这里是标题Title of the paper}
\author{作者LeeeeeAndLv}
\date{}

\begin{document}
	\maketitle
	
	\section{前言Introduction}
	
	In \cite{Au1900}, Author1 considered the problem of ...
	
	\begin{equation*}
		x^2 + y^2 = z^2
	\end{equation*}
	
	\begin{equation}\label{eq:GouGuThm}
		x^2 + y^2 = z^2
	\end{equation}
	
	\[
	\left\{
	\begin{aligned}
		a x + b y = c, \\
		d x + e y = g.
	\end{aligned}
	\right.
	\]
	
	\begin{thm}
		This is the first theorem. If conditions $A$ holds, i.e. $x\in \mathcal{X}$, then
		\[
		\int_\Omega u + \sum_{i=1}^\infty a_i + \oint_l \sqrt{\frac{\sin x}{\tan y}} = \bar{M}, \qquad \lim_{i\to \infty} a_i = 0.
		\]
	\end{thm}
	
	\begin{rmk}
		This is the content of Remark 1. 
	\end{rmk}
	
	
	\section{国内外研究现状Preliminaries}
	
	\section{证明Proof of the main results}
	
	\section{结论Conclusions}
	
	\begin{thebibliography}{99}
		\bibitem{Au1900}Author1, Title, {\it Journal Name}, 1900, 100(3), 432--456.  
	\end{thebibliography}
	
\end{document}
